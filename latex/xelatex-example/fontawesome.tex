\documentclass[12pt]{article}
\usepackage{fontspec}
\usepackage{fontawesome}

\usepackage{hyperref}
\usepackage{xcolor}
\definecolor{linkcolor}{rgb}{0, 0.2, 0.6}
\hypersetup{colorlinks, breaklinks, urlcolor = linkcolor, linkcolor = linkcolor}

\setmainfont{Liberation Sans}

\newfontfamily{\faf}{FontAwesome}

\newcommand\faIcon[1]{{\faf #1\ }}

\newcommand\twitter{\faIcon{\char"F099}}
\newcommand\twitterSquare{\faIcon{\char"F081}}
\newcommand\homeSolid{\faIcon{\char"F015}}
\newcommand\pie{\faIcon{\char"F705}}
\newcommand\linkedin{\faIcon{\char"F0E1}}
\newcommand\github{\faIcon{\char"F09B}}

\title{Font Awesome}
\author{
  Steven Shaw
  Loves programming languages
}
\date{Nov, 2019}

\begin{document}

  \maketitle

  Some Font Awesome icons:

  Manual

  fa-twitter \twitter{}
  fa-twitter-square \twitterSquare{}
  fa-home-solid \homeSolid{}
  fa-pie \pie{}
  fa-linkedin \linkedin{}
  fa-github \github{}

  Render

  \href{https://twitter.com/steshaw}{\twitter @steshaw}
  \href{https://steshaw.org/}{\homeSolid steshaw.org}

  Package

  \faicon{home}
  \faicon{twitter}
  \faicon{github}
  \faStackOverflow{}
  \faicon{linkedin}
  \faicon{cloud}
  \faicon{code}
  \faicon{linux}
  \faicon{apple}
  \faicon{laptop}
  \faicon{laptopCode}
  \faicon{pie}

\end{document}
